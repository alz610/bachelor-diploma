\anonsection{Заключение}

Освоены пакеты программ для параллельных вычислений: вычислительная платформа для автоматического решения ДУЧП FEniCS, библиотека для векторных вычислений Numpy.

Разработаны прототипы программ для численного решения прямой задачи БКЗ для двух известных моделей прискважинной зоны.

Проведены расчеты и сопоставление с известными результатами в литературе. Показана применимость метода решения прямой задачи БКЗ с приемлимой точностью. В данном случае мы следили за двумя параметрами: гладкостью решения и сравнением "на глаз" с известной палеткой. Поскольку принятая методика решения обратной задачи БКЗ основано на использованием палеток на "сравнении на глаз", то такой критерий годится. Относительная ошибка примерно ${<10}$ \%.

Все вычисления произведены на компьютере: операционная система Ubuntu 18.04 LTS, процессор Intel Pentium 4415U 2.30 ГГц с 4 логическими процессорами
($1 \text{ физический процессор} \times 2 \text{ ядра в физическом процессоре} \times 2 \text{ потока в каждом ядре}$).

Необходим анализ погрешности метода для курса дальнейшей работы.

\clearpage
