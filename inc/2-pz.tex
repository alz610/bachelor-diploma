\let\Omega\varOmega

\section{Математическая постановка задачи}

Формализуем задачу, описанную в предыдущем пункте. Во-первых, поместим излучающий электрод в начало координат ${\bm x = 0}$ и введем обозначение $I$ для силы протекающего через него тока.

% <Здесь вывод задачи Дирихле для электрического потенциала. Вывод и математическая постановка для кажущегося сопротивления потенцил- и градиент-зондов. Полагаю, нужно хорошо разобрать информацию в источнике \cite{elec} и в отчете по БКЗ Магадеева.>



Электрический потенциал ${u = u(\bm x)}$ электрического поля удовлетворяет задаче Дирихле для уравнения Пуассона \cite[с. 67]{elec}:
\begin{alignat}{2}
\nabla \cdot (\nabla u(\bm x) / \rho(\bm x)) &= f(\bm x),\qquad && \bm x \in \Omega, \label{eq:poisson}\\
u(\bm x) &= 0, && \bm x \in \partial \Omega, \label{eq:dirichet}
\end{alignat}
где $\Omega$ --- бесконечная область пространства, ${\partial \Omega}$ --- граница области $\Omega$, точки которой находятся на бесконечности (${\bm x \in \partial \Omega: |\bm x| \rightarrow \infty}$), ${f = f(\bm x)}$ --- источниковый член.

Источниковый член в случае точечного источника потенциала в точке ${\bm x = 0}$:
\begin{equation}
f(\bm x) = -I \delta(\bm x),
\end{equation}
где $\delta(\bm x)$ -- обобщенная функция дельта-функция Дирака.


\subsection{Слабая постановка}

Достаточно гладкая функция $u$, удовлетворяющий обеим равенствам \eqref{eq:poisson} и \eqref{eq:dirichet}, известна как классическое решение для краевой задачи. Для задачи Дирихле, функция $u$ является классическим решением, только если она дважды непрерывно дифференцируемая в открытой области $\Omega$ ($u \in C^2(\Omega)$) и непрерывная в закрытой области $\Omega$ ($u \in C^0(\overline \Omega)$). В случаях областей с негладкими границами или функций, не являющихся гладкими, $f$ --- источниковых членов (к тому же функция УЭС $\rho$ негладкая), функция $u$, удовлетворяющая \eqref{eq:poisson}--\eqref{eq:dirichet}, может не быть гладкой (или обычной) достаточно, чтобы считалась классическим решением. Поскольку $f$ не является гладкой функцинй, то вторая производная решения $u$ не является гладкой, и следовательно ${u \notin C^2(\Omega)}$ и не существует классического решения. Для таких задач, которые возникают из вполне обоснованных математических моделей, необходима альтернативная запись краевых задач. Поскольку эта альтернативная запись менее ограничивает в плане допустимых входных данных, она названа слабой постановкой \cite[с. 14]{FEFIS}.

FEniCS основан на методе конечных элементов, который является общим и эффективным математическим механизмом для численного решения дифференциальных уравнений в частных производных. Отправной точкой для методов конечных элементов является ДУЧП, выраженное в вариационной форме. Опыт показывает, что возможно работать с FEniCS в качестве инструмента для решения ДУЧП, даже не имея глубоких знаний о методе конечных элементов, при условии, что есть помощь постановки ДУЧП в виде вариационной задачи.

Простой рецепт превращения ДУЧП в вариационную задачу --- умножить ДУЧП на функцию $v$, интегрировать полученное уравнение по области $\Omega$ и выполнить интегрирование по частям с производными второго порядка. Функция $v$, которая множит ДУЧП, называется тестовой функцией. Неизвестная функция $u$, подлежащая аппроксимации, называется пробной функцией. Термины пробная и тестовая функции также используются в программах FEniCS. Пробная и тестовая функции принадлежат к определенным так называемым функциональным пространствам, которые определяют свойства функций.

В данном случае, сначала умножим уравнение Пуассона на тестовую функцию $v$ и интегрируем по заданной области $\Omega$: 
\begin{equation} \label{eq:none1}
  - \int_\Omega \nabla \cdot (\nabla u / \rho)v \dx 
  = \int_\Omega f v \dx.
\end{equation}

Здесь мы обозначили через $\dx$ дифференциальный элемент для интегрирования по области $\Omega$. Позже мы обозначим через $\ds$ дифференциальный элемент для интегрирования по границе $\Omega$.

Как правило, когда получают вариационную постановку, избавляются от высших степеней производных. Здесь мы имеем производную второго порядка функции $u$, которая может быть переведена в производные первого порядка функций $u$ и $v$ посредством применения метода интегрирования по частям. Формула гласит
\begin{equation} \label{eq:IBP}
  \int_\Omega \nabla \cdot (\nabla u / \rho)v \dx
= \int_{\partial\Omega} \frac 1 \rho {\partial u\over\partial n}v \ds
- \int_\Omega \frac 1 \rho \nabla u\cdot\nabla v \dx,
\end{equation}
где $\frac{\partial u}{\partial n} = \nabla u \cdot n$ есть производная $u$ по направлению внешней нормали $n$ на границе.

Еще одной особенностью вариационных постановок является то, что тестовая функция должна зануляться на участках границы, где известно $u$. В нашей задаче, это значит, что $v = 0$ на всей границе $\partial\Omega$. Первый член в правой части уравнения следовательно пропадает. Из \eqref{eq:none1} и \eqref{eq:IBP} следует
\begin{equation} \label{eq:weakform}
- \int_\Omega \frac 1 \rho \nabla u\cdot\nabla v \dx = \int_\Omega f v \dx.
\end{equation}

Если мы потребуем, чтобы это уравнение выполнялось для всех тестовых функций $v$ в некотором подходящем пространстве $\hat V$, так называемом тестовом пространстве, то мы получаем четко определенную математическую задачу, которая однозначно определяет решение $u$, которое лежит в некотором (возможно, другом) функциональном пространстве $V$, так называемое пробное пространство. Мы обозначим \eqref{eq:weakform} как слабую или вариационную форму исходной краевой задачи \eqref{eq:poisson} - \eqref{eq:dirichet}.

Правильная постановка нашей вариационной задачи теперь выглядит следующим образом: найти $u \in V$ такой, что
\begin{equation} \label{eq:varprob}
- \int_\Omega \frac 1 \rho \nabla u\cdot\nabla v \dx = \int_\Omega f v \dx.
\end{equation}

Пробные и тестовые пространства $V$ и $\hat V$ в данной задаче определены как
\begin{align}
V &= \left\{v \in H^{1}(\Omega): v=0 \text { on } \partial \Omega\right\}, \\
\hat{V} &= \left\{v \in H^{1}(\Omega): v=0 \text { on } \partial \Omega\right\}.
\end{align}

Область определения функций пространств есть $\Omega$,
и эти функции удовлетворяют граничным условиям:
значения функций на $\partial\Omega$ равны нулю.

В кратце, $H^1(\Omega)$ --- математически известное пространство Соболева, содержащее такие функции $v$, что $v^2$ и $|\nabla v|^2$ имеют конечные интегралы по $\Omega$ (по сути это означает, что функции непрерывны). Решение лежащего в основе ДУЧП должно лежать в функциональном пространстве, где производные также являются непрерывными, но пространство Соболева $H^1(\Omega)$ допускает функции с разрывными производными. Это более слабое требование непрерывности $u$ в вариационном утверждении \eqref{eq:varprob}, как результат интегрирования по частям, имеет большие практические последствия, когда речь идет о построении пространств функций конечных элементов. В частности, он позволяет использовать кусочно-полиномиальные функциональные пространства; то есть функциональные пространства, построенные путем сшивания полиномиальных функций в простых областях, таких как интервалы, треугольники или тетраэдры.


\subsection{Аппроксимация}

Вариационная задача \eqref{eq:varprob} является непрерывной задачей: она определяет решение $u$ в бесконечномерном функциональном пространстве $V$. Метод конечных элементов для уравнения Пуассона находит приближенное решение вариационной задачи \eqref{eq:varprob} путем замены бесконечномерных функциональных пространств $V$ и $\hat V$ на дискретные (конечномерные) пробное и тестовые функциональные пространства $V_h \subset V$ и $\ hat V_h \subset \hat V$ \cite[с. 14]{FEniCS}. Дискретная вариационная задача гласит: найдите $u_h \in V_h \subset V$, для которого
\begin{equation} \label{eq:varprob_approx}
\int_{\Omega} \nabla u_{h} \cdot \nabla v \mathrm{d} x=\int_{\Omega} f v \mathrm{d} x \quad \forall v \in \hat{V}_{h} \subset \hat{V}.
\end{equation}

Эта вариационная задача вместе с подходящим определением функциональных пространств $V_h$ и $\hat V_h$ однозначно определяют наше приближенное численное решение уравнения Пуассона \eqref{eq:poisson}. Заметим, что граничные условия записаны в части определения пробного и тестового пространств. Математическая структура может показаться сложной на первый взгляд, но хорошая новость заключается в том, что конечно-элементная вариационная задача \eqref{eq:varprob_approx} выглядит так же, как непрерывная вариационная задача \eqref{eq:varprob}, и FEniCS может автоматически решать вариационную задачу такие проблемы, как \eqref{eq:varprob_approx}.

Приближенное решение разыскивается в виде:
\begin{equation}
u_h = \sum_{i=1}^N c_i \varphi_i(\bm x)
\end{equation}
где $\varphi_1(\bm x), \varphi_2(\bm x), ..., \varphi_N(\bm x)$
--- некоторые заданные (базисные) функции,
удовлетворящие граничным условиям;
$c_1, c_2, ..., c_N$ --- неизвестные, подлежащие вычислению.
Их находят, решая систему алгебраических уравнений \cite[с. 14]{FEM_intro}:
\begin{equation}
\int_\Omega \frac 1 \rho \nabla u_h \cdot \nabla \varphi_i \dx = \int_{\Omega} f v \mathrm{d} x, \quad i = 1, 2, ..., N.
\end{equation}

$\varphi_i(\bm x)$ --- базисная функция, непрерывная на $\Omega$,
равная нулю на $\partial \Omega$ и заданная таким образом:
\begin{equation}
\varphi_i(\bm x_j) =
\left\{ \begin{aligned}
    0, \quad \text{если} \ \bm x_i \neq \bm x_j, \\
    1, \quad \text{если} \ \bm x_i = \bm x_j;
\end{aligned} \right.
\end{equation}
где $\bm x_i$ --- координаты $i$-того узла сетки.

\clearpage