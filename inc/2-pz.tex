\let\Omega\varOmega

\section{Математическая постановка задачи}

Формализуем задачу, описанную в предыдущем пункте. Во-первых, поместим
излучающий электрод в начало координат ${\bm x = 0}$ и введем обозначение $I$ для силы
протекающего через него тока.

<Здесь вывод задачи Дирихле для электрического потенциала. Вывод и математическая постановка для кажущегося сопротивления потенцил- и градиент-зондов. Полагаю, нужно хорошо разобрать информацию в источнике \cite{elec} и в отчете по БКЗ Магадеева.>

Электрический потенциал ${u = u(\bm x)}$ электрического поля удовлетворяет задаче Дирихле для уравнения Пуассона \cite[с. 67]{elec}:
\begin{alignat}{2}
\nabla \cdot (\nabla u(\bm x) / \rho(\bm x)) &= f(\bm x),\qquad && \bm x \in \Omega, \label{eq:poisson}\\
u(\bm x) &= 0, && \bm x \in \partial \Omega, \label{eq:dirichet}
\end{alignat}
где $\Omega$ --- бесконечная область пространства,
${\partial \Omega}$ --- граница области $\Omega$,
точки которой находятся на бесконечности (${\bm x \in \partial \Omega: |\bm x| \rightarrow \infty}$),
${f = f(\bm x)}$ --- источниковый член.

Источниковый член в случае точечного источника потенциала в точке ${\bm x = 0}$:
\begin{equation}
f(\bm x) = -I \delta(\bm x),
\end{equation}
где $\delta(\bm x)$ -- обобщенная функция дельта-функция Дирака.


\subsection{Слабая постановка}

Достаточно гладкая функция $u$, удовлетворяющий обеим равенствам \eqref{eq:poisson} и \eqref{eq:dirichet}, известна как классическое решение для краевой задачи. Для задачи Дирихле, функция $u$ является классическим решением, только если она дважды непрерывно дифференцируемая в открытой области $\Omega$ ($u \in C^2(\Omega)$) и непрерывная в закрытой области $\Omega$ ($u \in C^0(\overline \Omega)$). В случаях областей с негладкими границами или функций, не являющихся гладкими, $f$ --- источниковых членов (к тому же функция УЭС $\rho$ негладкая), функция $u$, удовлетворяющая \eqref{eq:poisson}--\eqref{eq:dirichet}, может не быть гладкой (или обычной) достаточно, чтобы считалась классическим решением. Поскольку $f$ не является гладкой функцинй, то вторая производная решения $u$ не является гладкой, и следовательно ${u \notin C^2(\Omega)}$ и не существует классического решения. Для таких задач, которые возникают из вполне обоснованных математических моделей, необходима альтернативная запись краевых задач. Поскольку эта альтернативная запись менее ограничивает в плане допустимых входных данных, она названа слабой постановкой \cite[с. 14]{FEFIS}.

<Текст на английском языке будет переден на русский язык.>

FEniCS is based on the finite element method, which is a general and efficient
mathematical machinery for the numerical solution of PDEs. The starting
point for the finite element methods is a PDE expressed in variational form.
Experience shows that you can work with FEniCS as
a tool to solve PDEs even without thorough knowledge of the finite element
method, as long as you get somebody to help you with formulating the PDE
as a variational problem \cite[с. 12]{FEniCS}.

The basic recipe for turning a PDE into a variational problem is to multiply
the PDE by a function $v$, integrate the resulting equation over the domain $\Omega$,
and perform integration by parts of terms with second-order derivatives. The
function $v$ which multiplies the PDE is called a test function. The unknown
function $u$ to be approximated is referred to as a trial function. The terms
trial and test functions are used in FEniCS programs too. The trial and
test functions belong to certain so-called function spaces that specify the
properties of the functions.

В данном случае, сначала умножим уравнение Пуассона на тестовую функцию $v$ и интегрируем по заданной области $\Omega$: 
\begin{equation} \label{eq:none1}
  - \int_\Omega \nabla \cdot (\nabla u / \rho)v \dx 
  = \int_\Omega f v \dx.
\end{equation}

We here let $\dx$ denote the differential element for integration over the domain
$\Omega$. We will later let $\ds$ denote the differential element for integration over
the boundary of $\Omega$.

Как правило, когда получают вариационную постановку, избавляются от высших степеней производных.
Здесь мы имеем производную второго порядка функции $u$, которая может быть переведена в производные первого порядка функций $u$ и $v$ посредством применения метода интегрирования по частям.
Формула гласит
\begin{equation} \label{eq:IBP}
  \int_\Omega \nabla \cdot (\nabla u / \rho)v \dx
= \int_{\partial\Omega} \frac 1 \rho {\partial u\over\partial n}v \ds
- \int_\Omega \frac 1 \rho \nabla u\cdot\nabla v \dx,
\end{equation}
где $\frac{\partial u}{\partial n} = \nabla u \cdot n$ есть производная $u$ по направлению внешней нормали $n$ на границе.

Еще одной особенностью вариационных постановок является то, что тестовая функция должна зануляться на участках границы, где известно $u$.
В нашей задаче, это значит, что $v = 0$ на всей границе $\partial\Omega$.
Первый член в правой части уравнения следовательно пропадает.
Из \eqref{eq:none1} и \eqref{eq:IBP} следует
\begin{equation} \label{eq:weakform}
- \int_\Omega \frac 1 \rho \nabla u\cdot\nabla v \dx = \int_\Omega f v \dx.
\end{equation}

If we require that this equation holds for all test functions $v$ in some suitable space $\hat V$, the so-called test space, we obtain a well-defined mathematical
problem that uniquely determines the solution $u$ which lies in some (possibly different) function space $V$, the so-called trial space. We refer to \eqref{eq:weakform} as
the weak form or variational form of the original boundary-value problem
\eqref{eq:poisson}--\eqref{eq:dirichet}.

The proper statement of our variational problem now goes as follows: find
$u \in V$ such that
\begin{equation} \label{eq:varprob}
- \int_\Omega \frac 1 \rho \nabla u\cdot\nabla v \dx = \int_\Omega f v \dx.
\end{equation}

The trial and test spaces $V$ and $\hat V$ are in the present problem defined as
\begin{align}
V &= \left\{v \in H^{1}(\Omega): v=0 \text { on } \partial \Omega\right\}, \\
\hat{V} &= \left\{v \in H^{1}(\Omega): v=0 \text { on } \partial \Omega\right\}.
\end{align}

Область определения функций пространств есть $\Omega$,
и эти функции удовлетворяют граничным условиям:
значения функций на $\partial\Omega$ равны нулю.
In short, $H^{1}(\Omega)$ is the mathematically well-known Sobolev space containing
functions $v$ such that $v^2$ and $|\nabla v|^2$ have finite integrals over $\Omega$
(essentially meaning that the functions are continuous). The solution of the underlying
PDE must lie in a function space where the derivatives are also continuous,
but the Sobolev space $H^{1}(\Omega)$ allows functions with discontinuous derivatives.
This weaker continuity requirement of $u$ in the variational statement \eqref{eq:varprob},
as a result of the integration by parts, has great practical consequences when it
comes to constructing finite element function spaces. In particular, it allows
the use of piecewise polynomial function spaces; i.e., function spaces constructed by stitching together polynomial functions on simple domains such
as intervals, triangles, or tetrahedrons.


\subsection{Аппроксимация}

<Текст на английском языке будет переден на русский язык.>

The variational problem \eqref{eq:varprob} is a continuous problem: it defines the solution u in the infinite-dimensional function space $V$. The finite element method
for the Poisson equation finds an approximate solution of the variational problem \eqref{eq:varprob} by replacing the infinite-dimensional function spaces $V$ and $\hat V$ by
discrete (finite-dimensional) trial and test spaces $V_h \subset V$ and $\hat V_h \subset \hat V$. The discrete variational problem reads: find $u_h \in V_h \subset V$ such that \cite[с. 14]{FEniCS}
\begin{equation} \label{eq:varprob_approx}
\int_{\Omega} \nabla u_{h} \cdot \nabla v \mathrm{d} x=\int_{\Omega} f v \mathrm{d} x \quad \forall v \in \hat{V}_{h} \subset \hat{V}.
\end{equation}

This variational problem, together with a suitable definition of the function spaces $V_h$ and $\hat V_h$, uniquely define our approximate numerical solution
of Poisson’s equation \eqref{eq:poisson}. Note that the boundary conditions are encoded
as part of the trial and test spaces. The mathematical framework may seem
complicated at first glance, but the good news is that the finite element variational problem \eqref{eq:varprob_approx} looks the same as the continuous variational problem
\eqref{eq:varprob}, and FEniCS can automatically solve variational problems like \eqref{eq:varprob_approx}!

Приближенное решение разыскивается в виде:
\begin{equation}
u_h = \sum_{i=1}^N c_i \varphi_i(\bm x)
\end{equation}
где $\varphi_1(\bm x), \varphi_2(\bm x), ..., \varphi_N(\bm x)$
--- некоторые заданные (базисные) функции,
удовлетворящие граничным условиям;
$c_1, c_2, ..., c_N$ --- неизвестные, подлежащие вычислению.
Их находят, решая систему алгебраических уравнений \cite[с. 14]{FEM_intro}:
\begin{equation}
\int_\Omega \frac 1 \rho \nabla u_h \cdot \nabla \varphi_i \dx = \int_{\Omega} f v \mathrm{d} x, \quad i = 1, 2, ..., N.
\end{equation}

$\varphi_i(\bm x)$ --- базисная функция, непрерывная на $\Omega$,
равная нулю на $\partial \Omega$ и заданная таким образом:
\begin{equation}
\varphi_i(\bm x_j) =
\left\{ \begin{aligned}
    0, \quad \text{если} \ \bm x_i \neq \bm x_j, \\
    1, \quad \text{если} \ \bm x_i = \bm x_j;
\end{aligned} \right.
\end{equation}
где $\bm x_i$ --- координаты $i$-того узла сетки.

\clearpage