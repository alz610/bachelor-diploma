\section{Расчетная сетка}

При применении метода конечных элементов к данной задаче возникают затруднения, связанные с тем,
что искомый потенциал поля имеет особенность (уходит на бесконечность) в окрестности источника тока.
Это приводит к необходимости сгущения расчётной сетки в этой окрестности, но сильное сгущение
приводит к чрезмерному росту числа узлов сетки и, как следствие, сильному росту времени, требуемому на решение.

Поэтому представляет актуальность задача построения оптимальной сетки, в которой сочетаются высокая точность
получаемого решения и малое число узлов.

\subsection{Постановка задачи}

В качестве модельной рассматривалась задача для трёхслойной осесимметричной модели системы скважина--пласт:
задача рассматривается в цилиндрической системе координат $(r, z)$, где $r > 0$ --- радиальная координата,
$z$ --- осевая координата, по которой область однородна (единственный однородный пласт бесконечной мощности),
первый слой --- сама скважина ($0 < r < 1$) с УЭС $\rho_\text с = 1$, второй слой --- зона проникновения ($1<r<r_\text{зп} = 9$)
с УЭС $\rho_{зп}=15$, третий слой --- пласт ($r>r_\text{п}$) с УЭС $\rho_\text{п}=10$. В начале координат размещается
источник тока силой 1.

В качестве основы для расчётной сетки использовалась одномерная сетка, узлы которой рассчитаны в параграфе
\ref{OneDim}, отложенная на лучах, исходящих из начала координат под различными углами.
Затем эта основа дополнялась узлами, лежащими на границах разрыва УЭС $r=1$ и $r=r_\text{п}$.
Также производилось дополнительное адаптивное сгущение сетки в областях, в которых наблюдалось сильное искажение
решения по сравнению с эталонным (полученным на сетке с большим числом узлов во всей области).

\subsection{Построение одномерной сетки}
\label{OneDim}

Задача БКЗ моделируется, как правило, для осесимметричного случая, что приводит к двумерным задачам.
Известно, что точное решение (потенциал электрического поля) для случая одинакового во всём пространстве
удельного электрического сопротивления (УЭС) есть константа, делённая на расстояние до источника:

$$\varphi(r, z)=\frac{C}{\sqrt{r^2+z^2}}$$ (здесь начало координат совпадает с источником тока)

Поскольку для любого распределения УЭС решение в малой окрестности источника тока практически определяется
УЭС в этой же окрестности (а она считается постоянной, равной УЭС жидкости в скважине), решение будет всегда
иметь одну и ту же особенность в начале координат и изменяться плавно вдали от источника тока.
В качестве первого приближения к оптимальной сетке можно использовать оптимальную одномерную сетку
для известного распределения $y=1/x$, и затем эту одномерную сетку распространять на несколько лучей,
исходящих из начала координат.

Поэтому первой поставленной задачей было построение оптимальной одномерной сетки для интерполяции
известной функции $y=1/x$:

{\bf Задача}. Дана функция $y=1/x$ на интервале $[\delta, +\infty)$ ($\delta > 0$). Построить на ней сетку
$\{x_i,\,i=0,1,\dots\}$ из наименьшего числа узлов (начиная с $x_0=\delta$) так, чтобы разность между
функцией $y$ и приближающим её сплайном первого порядка на этой сетке не превышала заданного значения
$\varepsilon$.

%Тут краткий обзор, что использовалось для решения, формулы, программы.
%URL-ссылка офомляется так: \cite{Python}
\def\ibreak{& \\ &}
\def\iline{& \\ \cline{1-2} &}

\makeatletter
\newenvironment{customenv}
  {\align &} % \start@align\@ne\st@rredtrue\m@ne
  {& \endalign}

\newenvironment{systemed}
  {\left\{ \begin{aligned} &}
  {& \end{aligned} \right.}

\newenvironment{totalited}
  {\left[ \begin{aligned} &}
  {& \end{aligned} \right.}


Математическая постановка:
\begin{customenv}
  f(x) = \frac 1 x \text{ --- исходная функция}
  \nonumber
  \ibreak
  X = ( x_i ) \, , \, i \in [0,N] \text{ --- сетка}
  \nonumber
  \ibreak
  x_0 = \delta \text{ --- известное}
  \nonumber
\end{customenv}
\begin{customenv}
  F(x) = \begin{totalited}
    F_0(x) \, , && x \in [x_0, x_1]
    \ibreak
    F_1(x) \, , && x \in [x_1, x_2]
    \ibreak
    \dots &&
    \ibreak
    F_{N-1}(x) \, , && x \in [x_{N-1}, x_N]
  \end{totalited}
  \ibreak
  F_i(x) = A_i \, x + B_i \, , \, x \in [x_i, x_{i+1}]
\end{customenv}
\begin{equation}
  \begin{systemed}
    F_i(x_i) = f(x_i)
    \ibreak
    F_i(x_{i+1}) = f(x_{i+1})
  \end{systemed}
  \Rightarrow
  A_i, B_i
\end{equation}
\begin{customenv}
  R(x) = \begin{totalited}
    R_0(x) \, , && x \in [x_0, x_1]
    \ibreak
    R_1(x) \, , && x \in [x_1, x_2]
    \ibreak
    \dots &&
    \ibreak
    R_{N-1}(x) \, , && x \in [x_{N-1}, x_N]
  \end{totalited}
  \ibreak
  R_i(x) = \left| F_i(x) - f(x) \right|, \, x \in [x_i, x_{i+1}]
\end{customenv}
\begin{customenv}
  R'_i(x) = 0 \, \Rightarrow \, x_\text{max}
  \ibreak
  R_i(x_\text{max}) = \varepsilon \, \Rightarrow \, x_{i+1}
\end{customenv}

Решение:
\begin{customenv}
  \begin{systemed}
    F_i(x_i) = f(x_i)
    \ibreak
    F_i(x_{i+1}) = f(x_{i+1})
  \end{systemed}
  \Rightarrow
  \begin{systemed}
    A_i \, x_i + B_i = f(x_i)
    \ibreak
    A_i \, x_{i+1} + B_i = f(x_{i+1})
  \end{systemed}
  \Rightarrow
  \nonumber
  \ibreak
  \begin{systemed}
    A_i = \frac {f(x_{i+1}) - f(x_i)} {x_{i+1} - x_i}
    \ibreak
    B_i = f(x_i) - A_i \, x_i
  \end{systemed}
\end{customenv}
\begin{customenv}
  R_i(x) = A_i \, x + B_i - \frac 1 x 
  \ibreak
  R'_i(x) = A_i + \frac 1 {x^2} = 0
  \, \Rightarrow \,
  x_\text{max} = \frac 1 {\sqrt{-A_i}}
  \ibreak
  R_i(x_\text{max}) =
  A_i \, \frac 1 {\sqrt{-A_i}} + B_i - \sqrt{-A_i} = \varepsilon
  \, \Rightarrow
  \label{eq:sympy}
\end{customenv}
\begin{equation}
  x_{i+1} = \begin{totalited}
    \frac{\varepsilon x_{i}^{2} + x_{i} - 2 \sqrt{\varepsilon x_{i}^{3}}}{\varepsilon^{2} x_{i}^{2} - 2 \varepsilon x_{i} + 1},
    && \text{ --- выражение для нового левого узла}
    \ibreak
    \frac{\varepsilon x_{i}^{2} + x_{i} + 2 \sqrt{\varepsilon x_{i}^{3}}}{\varepsilon^{2} x_{i}^{2} - 2 \varepsilon x_{i} + 1};
    && \text{ --- выражение для нового правого узла}
  \end{totalited}
\end{equation}

Решение уравнения \eqref{eq:sympy} было получено посредством библиотеки Sympy в Python.


\subsection{Построение двумерной сетки}

%Краткое описание экспериментов, ссылки на литературу, библиотеки (triangle и др.).

Для построения вершин шестиугольников использованы одномерные сетки: узлы решения для гиперболы
и узлы, равноотстоящие в логарифмическом масштабе.
Использована реализация триангуляции библиотеки triangle на Python c различными настройками:
режимы добавления новых узлов, учет графа, значения ограничения снизу значений углов треугольников.

\newcounter{exp}

\refstepcounter{exp} \theexp \label{text_fullmesh}.
Построены вершины шестиугольников из узлов решения для гиперболы.
Построены сегменты (отрезки) внутреннего и внешнего границ (сегменты изображены красными линиями)
и обозначена метка дыры (изображена красным крестиком) внутри внутренней границы для триангуляции (рис. \ref{fullmesh}).
Вызвана триангуляция библиотеки triangle с настройкой 'pq30',
где 'p' сообщает библиотеке о вводе графа, 'q30' --- ограничение снизу 30 градусов значений углов треугольников двухмерной сетки.

Кол-во узлов: 421

\begin{figure}[H]
  \includegraphics{exps/fullmesh_1}
\end{figure}
\begin{figure}[H]
  \includegraphics{exps/fullmesh_2}
  \caption{} \label{fig_fullmesh}
\end{figure}

\stepcounter{exp} \theexp.
Для расчетной сетки выбраны узлы, расположенные справа от оси $r = 0$,
и построены сегменты половинов внутреннего и внешнего границ.
Использованы лагранжевые элементы $\mathcal{P}_1$.

Кол-во узлов: 221

\begin{figure}[H]
  \includegraphics{exps/1_mesh}
\end{figure}
\begin{figure}[H]
  \includegraphics{exps/1_solve}
\end{figure}
\begin{figure}[H]
  \includegraphics{exps/1_rhog}
\end{figure}

\stepcounter{exp} \theexp.
Использованы узлы, равноотстоящие в логарифмическом масштабе.

Кол-во узлов: 1510

\begin{figure}[H]
  \includegraphics{exps/2_mesh}
\end{figure}
\begin{figure}[H]
  \includegraphics{exps/2_solve}
\end{figure}
\begin{figure}[H]
  \includegraphics{exps/2_rhog}
\end{figure}

\stepcounter{exp} \theexp.
Добавлены сегменты на местах разрыва коэффициента УЭС.

Кол-во узлов: 7201

\begin{figure}[H]
  \includegraphics{exps/3_mesh}
\end{figure}
\begin{figure}[H]
  \includegraphics{exps/3_solve}
\end{figure}
\begin{figure}[H]
  \includegraphics{exps/3_rhog}
\end{figure}

\stepcounter{exp} \theexp.
Использованы лагранжевые элементы $\mathcal{P}_3$.

Кол-во узлов: 7201

\begin{figure}[H]
  \includegraphics{exps/4_mesh}
\end{figure}
\begin{figure}[H]
  \includegraphics{exps/4_solve}
\end{figure}
\begin{figure}[H]
  \includegraphics{exps/4_rhog}
\end{figure}

\refstepcounter{exp} \theexp \label{optimal}.
Использован точечный источник вместо внутреннего граничного условия.

Кол-во узлов: 6659

\begin{figure}[H]
  \includegraphics{exps/5_mesh}
\end{figure}
\begin{figure}[H]
  \includegraphics{exps/5_solve}
\end{figure}
\begin{figure}[H]
  \includegraphics{exps/5_rhog}
\end{figure}

\stepcounter{exp} \theexp.
Добавлены на границе $r = 0$ узлы одномерной сетки.

Кол-во узлов: 6845

\begin{figure}[H]
  \includegraphics{exps/6_mesh}
\end{figure}
\begin{figure}[H]
  \includegraphics{exps/6_solve}
\end{figure}
\begin{figure}[H]
  \includegraphics{exps/6_rhog}
\end{figure}

\refstepcounter{exp} \theexp \label{ideal}.
Добавлены сегменты возле мест разрыва коэффициента УЭС. Использованы лагранжевые элементы $\mathcal{P}_2$.

Кол-во узлов: 156856

\begin{figure}[H]
  \includegraphics{exps/7_mesh}
\end{figure}
\begin{figure}[H]
  \includegraphics{exps/7_solve}
\end{figure}
\begin{figure}[H]
  \includegraphics{exps/7_rhog}
\end{figure}

<Здесь новые эксперименты, которые еще не оформлены,
и лучший эксперимент, в котором получилось приемлимое решение кажущегося сопротивления во всей области
с меньшим числом узлов.>

\subsection{Результаты}

Оптимальный результат получен в эксперименте X за время в среднем X сек.
со среднеквадратичным отклонением X сек. при N проходов с N циклами. Использованы:
функция PointSource библиотеки FEniCS для введения точечного источника потенциала; узлы, равноотстоящие в логарифмическом масштабе;
узлы лежащие в местах пересечений границ областей УЭС с концетрическими окружностями и лучами,
отрезки графа на местах разрыва коэффициента УЭС; лагранжевые элементы $\mathcal{P}_2$; настройка библиотеки triangle 'pq30'.
Использован ПК для вычисления: операционная система Ubuntu 18.04 LTS, процессор Intel Pentium 4415U 2.30 ГГц с 4 логическими процессорами
(4 = 1*2*2 --- 1 физический процессор, 2 ядра в физическом процессоре, 2 потока в каждом ядре).

В результате проведённых численных экспериментов разработан метод построения оптимальных сеток
для решения задачи БКЗ на сетке, позволяющий получить приемлимое численное решение с меньшими затратами ресурсов.

\clearpage