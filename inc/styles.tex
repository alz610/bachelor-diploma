%%% Таблицы %%%
\DeclareCaptionLabelSeparator{tabsep}{\tablabelsep} % нумерация таблиц
\DeclareCaptionFormat{split}{\splitformatlabel#1\par\splitformattext#3}

\captionsetup[table]{
        format=\tabformat,                % формат подписи (plain|hang)
        font=normal,                      % нормальные размер, цвет, стиль шрифта
        skip=.0pt,                        % отбивка под подписью
        parskip=.0pt,                     % отбивка между параграфами подписи
        position=above,                   % положение подписи
        justification=\tabjust,           % центровка
        indent=\tabindent,                % смещение строк после первой
        labelsep=tabsep,                  % разделитель
        singlelinecheck=\tabsinglecenter, % не выравнивать по центру, если умещается в одну строку
}

%%% Рисунки %%%
\DeclareCaptionLabelSeparator{figsep}{\figlabelsep} % нумерация рисунков

\captionsetup[figure]{
        format=plain,                     % формат подписи (plain|hang)
        font=normal,                      % нормальные размер, цвет, стиль шрифта
        skip=.0pt,                        % отбивка под подписью
        parskip=.0pt,                     % отбивка между параграфами подписи
        position=below,                   % положение подписи
        singlelinecheck=true,             % выравнивание по центру, если умещается в одну строку
        justification=centerlast,         % центровка
        labelsep=figsep,                  % разделитель
}

%%% Подписи подрисунков %%%
\DeclareCaptionSubType{figure}
\renewcommand\thesubfigure{\asbuk{subfigure}} % нумерация подрисунков
\DeclareCaptionFont{norm}{\fontsize{14pt}{16pt}\selectfont}

\captionsetup[subfloat]{
        labelfont=norm,                 % нормальный размер подписей подрисунков
        textfont=norm,                  % нормальный размер подписей подрисунков
        labelsep=space,                 % разделитель
        labelformat=brace,              % одна скобка справа от номера
        justification=centering,        % центровка
        singlelinecheck=true,           % выравнивание по центру, если умещается в одну строку
        skip=.0pt,                      % отбивка над подписью
        parskip=.0pt,                   % отбивка между параграфами подписи
        position=below,                 % положение подписи
}


% \usepackage{listings} % Оформление исходного кода
% \lstset{
%     basicstyle=\small\ttfamily, % Размер и тип шрифта
%     breaklines=true, % Перенос строк
%     tabsize=2, % Размер табуляции
%     literate={--}{{-{}-}}2, % Корректно отображать двойной дефис
%     keepspaces=true
% }

% Ширина текста минус ширина надписи 999
\newlength{\twless}
\newlength{\lmarg}
\setlength{\lmarg}{\widthof{999}}   % ширина надписи 999
\setlength{\twless}{\textwidth-\lmarg}

\lstset{ %
%    language=R,                     %  Язык указать здесь, если во всех листингах преимущественно один язык, в результате часть настроек может пойти только для этого языка
    numbers=left,                   % where to put the line-numbers
    numberstyle=\fontsize{12pt}{14pt}\selectfont\color{Gray},  % the style that is used for the line-numbers
    firstnumber=1,                  % в этой и следующей строках задаётся поведение нумерации 5, 10, 15...
    stepnumber=1,                   % the step between two line-numbers. If it's 1, each line will be numbered
    numbersep=5pt,                  % how far the line-numbers are from the code
    backgroundcolor=\color{white},  % choose the background color. You must add \usepackage{color}
    showspaces=false,               % show spaces adding particular underscores
    showstringspaces=false,         % underline spaces within strings
    showtabs=false,                 % show tabs within strings adding particular underscores
    frame=leftline,                 % adds a frame of different types around the code
    rulecolor=\color{black},        % if not set, the frame-color may be changed on line-breaks within not-black text (e.g. commens (green here))
    tabsize=2,                      % sets default tabsize to 2 spaces
    captionpos=t,                   % sets the caption-position to top
    breaklines=true,                % sets automatic line breaking
    breakatwhitespace=false,        % sets if automatic breaks should only happen at whitespace
%    title=\lstname,                 % show the filename of files included with \lstinputlisting;
    % also try caption instead of title
    basicstyle=\fontsize{12pt}{14pt}\selectfont\ttfamily,% the size of the fonts that are used for the code
%    keywordstyle=\color{blue},      % keyword style
    commentstyle=\color{ForestGreen}\emph,% comment style
    stringstyle=\color{Mahogany},   % string literal style
    escapeinside={\%*}{*)},         % if you want to add a comment within your code
    morekeywords={*,...},           % if you want to add more keywords to the set
    inputencoding=utf8,             % кодировка кода
    xleftmargin={\lmarg},           % Чтобы весь код и полоска с номерами строк была смещена влево, так чтобы цифры не вылезали за пределы текста слева
    keepspaces=true,
    literate={--}{{-{}-}}2, % Корректно отображать двойной дефис
}
