\anonsection{Список сокращений и условных обозначений} % Заголовок

\noindent
%\begin{longtabu} to \dimexpr \textwidth-5\tabcolsep {r X}
\begin{longtabu} to \textwidth {r X}
% Жирное начертание для математических символов может иметь
% дополнительный смысл, поэтому они приводятся как в тексте
% диссертации
$p$ & давление, бар \\
$r$ & радиус, м \\
$t$ & время, часы \\
$q$ & постоянный дебит в поверхностных условиях, $\text{м}^3 / \text{сутки}$ \\
$h$ & мощность пласта, м \\
$k$ & проницаемость, мДа \\
$\mu$ & вязкость, $\text{мПа} \cdot \text{с}$ \\
$c_t$ & сжимаемость, $\text{бар}^{-1}$ \\
$\phi$ & пористость, доли ед. \\
$s$ & скин-фактор \\
$B$ & объемный коэффициент \\
$C$ & коэффициент влияния объема ствола скважины, $\text{м}^3 / \text{бар}$ \\

\end{longtabu}
\addtocounter{table}{-1}% Нужно откатить на единицу счетчик номеров таблиц, так как предыдующая таблица сделана для удобства представления информации по ГОСТ

\clearpage
