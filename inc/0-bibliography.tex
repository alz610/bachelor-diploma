\begingroup 
\renewcommand{\section}[2]{\anonsection{Библиографический список}}
\begin{thebibliography}{00}

\bibitem{elec}
    Электрическое зондирование геологической среды. Часть 1. Прямые задачи и методика работ /
    И. Н. Модин, В. А. Шевнин, В. К. Хмелевской [и др.]
    --- МГУ Москва, 1988.
    --- 176 с.
    --- URL: \url{http://geophys.geol.msu.ru/STUDY/5KURS/Book1_1988MSU.pdf}
   
\bibitem{FEFIS}
    Finite Elements and Fast Iterative Solvers: with Applications in Incompressible Fluid Dynamics (2nd edition) /
    H. Elman, D. Silvester, A. Wathen.
    --- OUP, Oxford, 2014. 
    --- 480 с.
    --- URL: \url{https://books.google.ru/books?id=Ly-TAwAAQBAJ}

\bibitem{FEniCS}
    Solving PDEs in Python --The FEniCS Tutorial Volume I /
    H. P. Langtangen, A. Logg.
    --- Springer, 2017.
    --- 153 с.
    --- URL: \url{https://fenicsproject.org/pub/tutorial/pdf/fenics-tutorial-vol1.pdf}

\bibitem{FEM_intro}
    Введение в теорию метода конечных элементов. Учебное пособие /
    Р.З. Даутов, М.М. Карчевский.
    — Казань: Казанский государственный универcитет им. В.И. Ульянова–Ленина, 2004.
    — 239 с.
    --- URL: \url{https://kpfu.ru/staff_files/F1229619272/MKEbookDRZ.pdf}

\bibitem{palette}
    Технология создания многопараметричных
    палеток для решения прямых и обратных задач
    скважинной геоэлектрики /
    К. С. Сердюк [и др.] //
    Каротажник. Результаты исследований и работ ученых и конструкторов.
    --- 2014
    --- Вып. 241.
    --- С. 32-41.

\bibitem{valiullin}
    Геофизические исследования и работы в скважинах: в 7 т. Т. 1.
    Промысловая геофизика /
    / Р.А. Валиуллин, Л.Е. Кнеллер.
    --- Уфа: Информреклама.
    --- 2010.
    --- 172 с.

\end{thebibliography}
\endgroup

\clearpage
