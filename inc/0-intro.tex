\anonsection{Введение}

Проблема и ее актуальность. Геофизические исследования скважин
электрическими методами используются для
реконструкции пространственного распределения
УЭС, которое связано с нефтенасыщением.
Задачи электрокаротажа нелинейны, а обратная
задача определения параметров по измерениям
некорректна (неоднозначна). Решение задач в
полных постановках весьма ресурсоемко и плохо
подходит для промышленного применения.

Обработка диаграмм БКЗ заключается в выделении пластов
и отсчете существенных значений $\rho_\text {к}$ против них, построении
кривых зависимости $\rho_\text {к}$ от размера зонда --- кривых зондирования
и кривых БКЗ, сравнении полученных кривых с расчетными
для определения удельного сопротивления пластов и выявлении
зон проникновения фильтрата ПЖ в пласт \cite{valiullin}.

Основная идея палеточного подхода для решения задач скважинной
геоэлектрики заключается в задании функции связи входных
и выходных данных решаемой задачи на основе интерполяции по
дискретному множеству имеющихся решений.
Преимущества рассматриваемого
подхода состоят в отсутствии зависимости скорости
решения от входных данных и существенном снижении требований к
качеству начального приближения, как это требуется в итеративных алгоритмах
минимизации невязки теоретических и экспериментальных данных \cite{palette}.
Недостатком является больший объем данных, необходимых для хранения палетки.

В настоящее
время на практике используют палетки, как правило, для
двух- и трехслойных моделей, покрывающих наиболее
распространенные ситуации.

На практике возникает необходимость в палетках с такими параметрами, которые не опубликованы. Поэтому была поставлена задача решения прямой задачи БКЗ, где была бы возможность задавать необходимые параметры модели для построения палеток.

\clearpage