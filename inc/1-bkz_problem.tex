\section{Прямая задача БКЗ}

Боковое каротажное зондирование (БКЗ) состоит в измерении
потенциала электрического поля, создаваемого в среде точечным электрическим зарядом.
Далее по измеренному распределению потенциала предпринимается попытка воссоздать
геометрию среды, а также ее удельное сопротивление в разных точках пространства.

Поскольку наличие электрического поля в проводящей среде означает
возникновение в ней тока, точечный заряд, о котором говорилось выше, фактически
представляет собой электрод, находящийся под напряжением относительно бесконечно
удаленных точек (условимся называть его излучающим). Технически это достигается за
счет использования дополнительного электрода, располагаемого на большом расстоянии от
первого; тем самым цепь замыкается, и течение тока становится возможным. В
действительности нас не будет интересовать распределение поля на существенном
удалении от излучающего электрода, поэтому мы ограничимся моделью, в которой ток
растекается от него на условную бесконечность, обладающую нулевым потенциалом.
Следует отметить, что на практике как размер излучающего электрода, так и его потенциал
являются конечными. При этом, однако, становится необходимым детальный учет
геометрии области контакта прибора со средой. Дабы избежать подобных затруднений, мы
будем считать электрод точечным, а распределение поля вокруг него – сингулярным,
задавая интенсивность излучения не потенциалом излучающего электрода, а силой
протекающего через него тока.

Распределение потенциала в среде регистрируется приемными электродами,
наличие которых само по себе искажает линии тока в пространстве. Этим эффектом,
однако, можно пренебречь, если величина тока, протекающего через приемные электроды,
невелика. Далее мы будем считать, что последнее предположение верно.

Расположение приемных электродов варьируется в зависимости от того, какие
именно величины предполагается наблюдать непосредственно. Если речь идет об
измерении потенциала электрического поля как такового, то каждый отдельно взятый
электрод может рассматриваться как самостоятельный зонд; в этом случае мы имеем дело
с потенциал-зондом (строго говоря, техническая реализация потенциал-зонда может быть
более сложной, но для теоретических расчетов это значения не имеет). Если же измерять
предполагается не сам потенциал, а составляющую его градиента (т.е., с точностью до
знака, компоненту напряженности электрического поля), используется пара близко
расположенных приемных электродов - так называемый градиент-зонд. При таком подходе
рассматриваются не потенциалы электродов в отдельности, а разность их потенциалов,
которая и позволяет судить об искомой величине градиента. Независимо от типа зонда, под
его длиной мы будем понимать расстояние между излучающим электродом и приемными.

На практике результаты измерений БКЗ принято представлять не значениями
потенциалов, которые пропорциональны силе тока в излучающем электроде, а
независящими от тока значениями кажущихся удельных сопротивлений. Под кажущимся
удельным сопротивлением понимается удельное сопротивление такой однородной среды,
3в которой измерение данным зондом привело бы в точности к тем же результатам, что были
получены фактически. Таким образом, при исследовании однородной среды методом БКЗ
все зонды, независимо от их типов и длин, приводили бы к одному и тому же значению
кажущегося удельного сопротивления, которое соответствовало бы истинному удельному
сопротивлению вещества, заполняющего пространство. В случае неоднородной среды
никакой прямой аналогии нет, и вычисление ее истинных параметров по кажущимся
удельным сопротивлениям становится довольно сложной задачей, которая и представляет
собой обратную задачу БКЗ. На практике она может быть сформулирована следующим
образом: даны значения кажущегося удельного сопротивления, полученные с нескольких
разных зондов (отличающихся, как правило, длиной, но также, возможно, типом и
расположением); найти удельное сопротивление среды в каждой точке пространства. Ясно,
что в отсутствие дополнительных упрощающих предположений эта задача становится
принципиально неразрешимой.

Прямая задача БКЗ заключается в
прогнозировании кажущихся удельных сопротивлений, которые должны быть получены в
процессе измерения зондами заданных типов и длин. Характеристики сред, в которых
осуществляются измерения, также будем считать известными. Прямая задача, в отличие от
обратной, всегда разрешима.

\clearpage
